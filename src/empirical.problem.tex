\section{Research Problems and Hypotheses}

Our main research question deals with how a specific social navigation
technique, activity streams, could potentially influence usage of a web site:

\begin{quote}
  \rp[0]{
    Can social navigation through activity streams influence
    usage of an established web site?
  }
\end{quote}

From this main research question we also proposed more specific problem
statements that more clearly states different ways of influencing usage:

\begin{quote}
  \rp[1]{
    Do users perceive social navigation through activity streams as helpful in
    order to keep up-to-date on favorites' activities on \urort{}?
  }
\end{quote}

This question deals with the way users keep up-to-date on what their favorites
are doing on \urort{}. We want to investigate if activity streams can help
users in this task.

We were also concerned with how activity streams influenced the
frequency of keeping up-to-date on activities:

\begin{quote}
  \rp[2]{
    Does social navigation through activity streams lead users to more often
    keep up-to-date on favorites' activities on \urort{}?
  }
\end{quote}

The next research question deals with with how activity streams could
potentially influence the importance of favorites on \urort{}:

\begin{quote}
  \rp[3]{
    Does social navigation through activity streams lead users to make
    more artists on \urort{} their favorites?
  }
\end{quote}

We also had a more technical research question relating to how one can
conduct experiments on established web sites with Greasemonkey:

\begin{quote}
  \rp[4]{
    Can prototyping with Greasemonkey be considered a
    viable technical option when testing user behavior in an
    established web site?
  }
\end{quote}

This question seek to investigate whether Greasemonkey prototyping should be
considered as one of potentially many technical alternatives in future
research experiments, where user behavior is tested. As this is a new way
to test user behavior in relation to social navigation, we're mainly concerned
with gaining experience with using such a technical solution.

\parabreak

We created hypotheses for the majority of our research questions. After having
presented our results in 
\sectionref{empirical.results} we'll test these hypotheses in
\sectionref{empirical.discussion}.

\addline

Our $H1$ hypothesis deals with how easy respondents can keep-up-to
date with favorites' activities with and without an 
activity stream.
$B$ is the degree respondents can easily keep up-to-date on
favorites' activities without an activity stream. $A$ is the degree
respondents can easily keep-up-to-date on favorites' activities  with
an activity stream.
\begin{items}
  \iterm{$H1_0$:} $B \geq A$
  \iterm{$H1_A$:} $B < A$
\end{items}

Relating to the hypotheses about the degree respondents can keep
up-to-date on activities we have a $H2$ hypothesis concerning
the frequency of keeping up-to-date.
$B$ is the frequency respondents keep up-to-date on favorites' activities
without an activity stream. $A$ is the frequency respondents keep
up-to-date on favorites' activities with an activity stream.
\begin{items}
  \iterm{$H2_0$:} $B \geq A$
  \iterm{$H2_A$:} $B < A$
\end{items}

A $h3$ hypothesis about how activity streams influences favorite usage is
stated next.
$B$ is the amount of favorites for respondents without an activity stream.
$A$ is the amount of favorites for respondents having used an activity
stream.
\begin{items}
  \iterm{$H3_0$:} $B \geq A$
  \iterm{$H3_A$:} $B < A$
\end{items}
