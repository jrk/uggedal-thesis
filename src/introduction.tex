\chapter{Introduction}

% The point of the introduction is to answer: what is this thesis about?
% Explain this in four steps by:
%
%   * Why you have choosen this topic rather than any other. Examples:
%       - has been neglected
%       - much discussed but not properly and fully
%   * Why this topic interests you.
%   * The kind of research approach or academic disciple you will utilize.
%   * Your research questions or problems.
%
% The role of the introduction, like the abstract, is to orient your readers.
% This is best done clearly and succintly.
%
% This structure is frequently seen in other UiO thesises:
% 
%   * Motivation
%   * Objective
%   * Contributions
%   * Structure of Thesis
%
% Some also have a methodology section in the introduction instead of a
% seperate chapter:
% 
%   * Motivation
%   * Problem Statement
%   * Research Methodology
%   * Scope of Thesis
%   * Summary of Results
%   * Thesis Outline

The web is becoming increasingly more social as it's becoming more ordinary to
take part in some sort of web community. At the same time technological
advances and maturity have made it easy and cheap to create new communities.
This means that we're seeing an aboundance of such communities appear.
They're all competing to establich self sufficient user bases, and if the
objective is to achive large profits, to adopt as many users
as possible. This creates a strong incentive to come up with interesting new
features and best your competitors when it comes to pleasing your users.
As a result we're seeing rapid innovation in this area of the web infamously
coined ``Web 2.0''
\footnote{Web 2.0 was first used as the name of a conference arranged by
O'Reilly Media. The ``2.0'' part of the conference name was then used to
signify the revival of interest in the web after the dot-com bubble in the
early 21st century \citep{oreilly07}.
Later the founder of O'Reilly Media, Tim O'Reilly, defined
the term as the characteristics of the web services that survived the dot-com
bubble and the web services he deemed to be the best newcommers to the
field \citep{oreilly05}.}.

This thesis focuses on navigational problems and only those which are of a
social nature. Having efficient and easy to use navigation is clearly
essential for serving your users' intrests. As the content on the web to a
higher degree than before are user generated it has become almost impossible
for the creators of a given web service to design sound navigation without
relying on your loyal users. Insteaed such structures have to be organically
grown by designing navigational schemes that harness the work your user base
is conducting as they use your service. This is the essence of social
navigation as the first definition of the term clearly and succinctly
captures:

\begin{quote}
In social navigation, movement from one item to another is provoked as an
artefact of the activity of another or a group of others. \citep{dourish94} 
\end{quote}

\section{Motication}

Social navigation is a well defined term within the academic
community and there have been several studies on it's applicabibity in context
of the web. Most of this research took place in the early days of the
web--preceding our current area of Web 2.0. As described earlier the
increase we're seeing in web services with social aspects are bound to provide
for innovations in the space of navigation. It would therefore be interesting
to look at some of the state-of-the-art social web services and look at what
contributions they have made to the field of social navigation.





% This thesis will describe how navigation is used in a social manner in
% modern web applications and arrive at a taxonomy of social navigation
% techniques.
%
% We are lacking material on how one can use social navigation conciously in
% the process of designing web applications for our modern web age. I will
% try to remedy this by showing results of implementation of some of the
% navigational techniques presented.
%
% The material presented will help guide design of social web applications.
%
% We lack collections of proven navigational design patterns that are social
%of nature.
