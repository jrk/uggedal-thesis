\chapter{Introduction}

% The point of the introduction is to answer: what is this thesis about?
% Explain this in four steps by:
%
%   * Why you have choosen this topic rather than any other. Examples:
%       - has been neglected
%       - much discussed but not properly and fully
%   * Why this topic interests you.
%   * The kind of research approach or academic disciple you will utilize.
%   * Your research questions or problems.
%
% The role of the introduction, like the abstract, is to orient your readers.
% This is best done clearly and succintly.
%
% This structure is frequently seen in other UiO thesises:
% 
%   * Motivation
%   * Objective
%   * Contributions
%   * Structure of Thesis
%
% Some also have a methodology section in the introduction instead of a
% seperate chapter:
% 
%   * Motivation
%   * Problem Statement
%   * Research Methodology
%   * Scope of Thesis
%   * Summary of Results
%   * Thesis Outline


% some even more introductury text before this perhaps?
As we'll see social navigation is a well defined term within the academic
community and there have been several studies on it's applicabibity in context
of the web. Most of this research took place in the early days of the
web--preceding our current area imfamously known as ``Web 2.0''. Since
there have been an enourmous increase in web applications with social ascpects
the later years it would be interesting to look at current practices and usage
of social navigation in some of the state-of-the-art social web applications.

This thesis will describe how navigation is used in a social manner in modern
web applications and arrive at a taxonomy of social navigation techniques.

We are lacking material on how one can use social navigation conciously in the
process of designing web applications for our modern web age. I will try to
remedy this by showing results of implementation of some of the navigational
techniques presented.

The material presented will help guide design of social web applications.

We lack
collections of proven navigational design patterns that are social of nature.

