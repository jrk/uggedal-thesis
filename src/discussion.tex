\chapter{Discussion}
\label{chapter:discussion}

This chapter will start with a synthesis based on the analysis of our
collected data on social navigation in modern web sites.
Next we'll discuss how our technique for transparently prototyping
applications relates to both developers and users. Lastly we'll look at how
activity streams fares as a social navigation technique.

\section{Social Navigation on the Social Web}

\section{Transparent Prototyping with Greasemonkey}

\subsection{Development perspective}

During our development of a prototype application with Greasemonkey for
enhancing an established web page we got a feel for its pros and cons from a
development perspective.

\subsubsection{Requires no access to the established implementation}

\subsubsection{Requires little knowledge of the established implementation}

\subsubsection{Requires more work than altering the established
  implementation}

\subsubsection{Fragile when the established implementation is changed}
% Only happened once during a two month span on \urort{}. What was scary
% thought was that the changes made the user script on the client side
% obsolete. If this had happened under production usage when the user scripts
% was pushed to the clients we would be in a world of trouble. Changes on the
% server side platform can be handled more transparently.

\subsubsection{Less performant than the established implementation}

\subsection{User perspective}

When we conducted a study of our prototype with real world users
we got valuable feedback on how well such a system works for the average user.

\subsubsection{Limited in browser selection}

\subsubsection{Difficulties with installing Greasemonkey}

\subsubsection{Difficulties with installing user-scripts}


\section{Activity Streams as a Social Navigation Technique}
